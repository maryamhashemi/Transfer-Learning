\begin{abstract}
تهیه داده‌های دارای برچسب برای آموزش الگوریتم‌های یادگیری ماشین بسیار مهم است. با این حال، بدست آوردن تعداد کافی از داده‌های برچسب‌دار در کاربردهای واقعی اغلب گران و زمانبر است. بنابراین استفاده از روش‌هایی که دانش موجود در یک دامنه با برچسب‌گذاری مناسب (دامنه مبدا) را به یک دامنه با برچسب‌های معدود یا بدون برچسب(دامنه هدف) منتقل می‌کنند، موثر هستند. به این روش‌ها انتقال یادگیری، تطبیق دامنه و یا تطبیق توزیع می‌گویند. از آنجا که دامنه‌های مبدا و هدف  دارای توزیع‌های مختلف هستند، روش‌های بیشماری برای کاهش واگرایی توزیع‌ها پیشنهاد شده است. در این پروژه به بررسی یک روش پارامتری به نام 
 \lr{\textit{BDA}}
 و  و یک روش غیرپارامتری به نام 
 \lr{\textit{EasyTL}}
  می‌پردازیم.
\end{abstract}