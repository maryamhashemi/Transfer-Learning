\section{شرح روش پیشنهادی مقالات}
{ 	
	در این بخش ابتدا مسئله تطبیق دامنه را به صورت ریاضی بیان می‌کنیم. سپس مشکلات و چالش‌هایی که در حل این مسئله وجود دارد را بیان خواهیم کرد. در ادامه نیز توضیح خواهیم داد که هر کدام از مقاله‌ها چه روشی را برای حل این مشکلات و چالش‌ها ارائه داده‌اند.
	
	\subsection{تعریف ریاضی مسئله تطبیق دامنه}
	{
		ما یک دامنه‌ی برچسب خورده به نام دامنه مبدا و یک دامنه برچسب نخورده به نام دامنه هدف داریم که به ترتیب به  صورت 
		$\left\{x_{s_i},\ y_{s_i}\right\}_{i=1} ^ n $
		و 
		$\left\{x_{t_j}\right\}_{j=1} ^ m$
		تعریف می‌شود. همچینین فرضیاتی را در مورد فضای ویژگی‌ها
		$X_s = X_t$
		و فضای برچسب‌ها
		$Y_s=Y_t$
		و احتمال حاشیه‌ای  
		${P_s(x}_s)\neq{P_t(x}_t)$
		و احتمال شرطی
		${P_s(y_s|x}_s)\neq{P_t(y_t|x}_t)$
		در نظر می‌گیریم. بنابراین هدف انتقال یادگیری این است که فضای برچسب‌های دامنه هدف 
		$Y_t$
		را به کمک دامنه مبدا
		$D_s$
		یاد بگیرد. تطبیق دامنه تلاش می‌کند که با کاهش واگرایی احتمال‌های حاشیه‌ای و احتمال‌های شرطی  بین دامنه‌ی مبدا و هدف، مسئله انتقال یاد گیری را حل کند به عبارت دیگر واگرایی بین 1)
		${P_s(x}_s)$
		و 
		${P_t(x}_t)$
		، 2) 
		${P_s(y_s|x}_s)$
		و
		${P_t(y_t|x}_t)$
		را به حداقل می‌رساند.
	}

	\subsection{تطبیق  متوازن دامنه در یادگیری انتقال
		\protect \footnote{\lr{Balanced Distribution Adaptation for Transfer Learning}}}
	{
		بسیاری از روش‌های تطبیق توزیع موجود، یکی از توزیع‌های حاشیه‌ای یا شرطی و یا هر دو را تطبیق می‌دهند. در مقاله‌های اخیر ثابت شده است که استفاده از هر دو توزیع عملکرد بهتری را نتیجه می‌دهد. اما روش‌های کنونی تاثیر هر دو توزیع را یکسان در نظر می‌گیرند در حالی که وقتی مجموعه داده‌ها بسیار متفاوت باشند، به این معنی است که توزیع حاشیه‌ای اهمیت بیشتری دارد و وقتی مجموعه داده‌ها مشابه هستند، به این معنی است که توزیع شرطی به توجه بیشتری نیاز دارد. از این رو، لازم است از هر دو توزیع با وزنی مناسب با اهمیت آن‌ها برای تطبیق استفاده نمود. علاوه بر این، نامتوازن بودن مجموعه‌داده بر روی تطبیق دامنه تاثیرگذار است که در روش‌های موجود این موضوع در نظر گرفته نمی‌شود. برای حل این دو مشکل، در
		\cite{wang2017balanced}
		دو روش پیشنهاد شده است. روش اول BDA است که نه تنها می‌تواند توزیع‌های حاشیه‌ای و شرطی بین حوزه‌ها را تطبیق دهد، بلکه اهمیت این دو توزیع را به صورت متوازن تنظیم می‌کند. روش دوم
		\lr{W-BDA}
		است که مشکل نامتوازن بودن مجموعه داده را حل می‌کند.
		\lr{W-BDA}
		می‌تواند هنگام انجام تطبیق توزیع، وزن هر کلاس را به طور انطباقی تغییر دهد.
	 	\subsubsection{BDA}
	 	{
 		همانطور که توضیح دادیم در تطبیق دامنه، با کاهش واگرایی یا فاصله ی بین احتمال های حاشیه ای و شرطی می توانیم دامنه های مبدا و هدف را تطبیق دهیم. این موضوع را میتوانیم به صورت عبارت زیر بیان کنیم:
 		\begin{equation}
 			Dist\left(D_{s,}D_t\right)\approx Dist{(P\ (x}_s),{P(x}_t))+Dist({P(y_s|x}_s),{P(y_t|x}_t))
 			\label{eq:1}
 		\end{equation}
 		روش BDA به عبارت 
 		\ref{eq:1}
 		 یک ضریب توازن به نام $\mu$ اضافه می کند که میزان اهمیت توزیع حاشیه ای و شرطی را در مسائل مختلف تنظیم و کنترل می کند. پس عبارت 
 		\ref{eq:1}
 		 تبدیل می شود به:
 		 \begin{equation}
 			Dist\left(D_{s,}D_t\right)\approx(1-\mu)Dist{(P\ (x}_s),{P(x}_t))+\mu Dist({P(y_s|x}_s),{P(y_t|x}_t))
 			\label{eq:2}
 		 \end{equation}
 		  برای حل عبارت 
 		  \ref{eq:2}
 		  از روش MMD برای تخمین توزیع های حاشیه ای و توزیع های شرطی استفاده می کنیم. بنابراین خواهیم داشت:
 		  \begin{equation}
 		  	 D \left( D_{s,}D_{t} \right)  \approx  \left( 1- \mu  \right)  \bigg \|\frac{1}{n} \sum _{i=1}^{n}x_{s_{i}} - \frac{1}{m} \sum _{j=1}^{m}x_{t_{j}}\bigg \| _{H}^{2} + \mu \sum _{c=1}^{C} \bigg \Vert \frac{1}{n_{c}} \sum _{x_{s_{i}} \in  D_{s}^{ \left( c \right) }}^{n}x_{s_{i}} - \frac{1}{m_{c}} \sum _{x_{t_{j}} \in  D_{t}^{ \left( c \right) }}^{n}x_{t_{j}} \bigg \Vert _{H}^{2}
 		  	 \label{eq:3}
 		  \end{equation}
 		  
 		  در عبارت 
 		 \ref{eq:3}
 		  ترم اول فاصله ی توزیع های حاشیه ای بین دامنه ها را نشان می دهد و ترم دو م هم  فاصله ی توزیع های شرطی بین دامنه ها را نمایش می دهد. با بهره گیری از تکنیک های ماتریس و جبر خطی عبارت 
 		 \ref{eq:3}
 		  را می توان به صورت زیر نوشت:
 		  \begin{equation}
	 	  \begin{aligned}
		  min \quad tr \left( A^{T}X \left( \left( 1 - \mu \right) M_{0} + \mu \sum _{c=1\mathrm{ }}^{C}M_{c} \right) X^{T}A \right) + \lambda \Vert A \Vert_{F}^{2} \\
		  s.t. \quad A^{T}XHX^{T}A = I, \quad  0  \leq \mu \leq  1 \quad \quad \quad \quad \quad
		  \label{eq:4}
	 	  \end{aligned}
 		  \end{equation}
 		   در عبارت
 		   \ref{eq:4}
 		    دو ترم وجود دارد: 1) تطبیق توزیع های حاشیه ای و شرطی به همراه ضریب توازن $\mu$ 2) ترم مربوط به رگلاریسیون با ضریب رگلارسیون $\lambda$. در این عبارت 2 قید هم داریم. قید اول بیان می کند که داده ی تغییر یافته ویژگی های داخلی اصل داده را حفظ می کند. قید دوم نیز بازه ی ضریب توازن را بیان می کند.
 		    
 		    برای حل عبارت 
 		    \ref{eq:4}
 		    از ضرائب لاگرانژ استفاده می کنیم. بنابراین اگر ضرائب لاگرانژ 
 		    $\phi = \left\lbrace \phi_1, \phi_2, ... , \phi_d \right\rbrace  $
 		    باشد، خواهیم داشت: 
 		    \begin{equation}
 		    	\begin{aligned}
 		    	L = tr \left( A^{T}X \left( \left( 1 - \mu \right) M_{0} + \mu \sum _{c=1\mathrm{ }}^{C}M_{c} \right) X^{T}A \right) + \lambda \Vert A \Vert_{F}^{2} + tr \left( \left( I -  A^{T}XHX^{T}A)\right) \phi \right) 
 		    	\label{eq:5}
 		    	\end{aligned}
 		    \end{equation}
 		    
 		    برای بدست آوردن ماتریس تبدیل $A$ کافی است از عبارت
 		    \ref{eq:5}
 		    نسبت به $َA$ مشتق بگیریم و برابر صفر بگذاریم. در این صورت به عبارت زیر دست می یابیم:
 		    \begin{equation}
 		    	\left( X \left( \left( 1 - \mu \right) M_{0} + \mu \sum _{c=1\mathrm{ }}^{C}M_{c} \right) X^{T} + \lambda I \right) A = XHX^{T}A \phi
 		    	\label{eq:6}
 		    \end{equation}
 		 	عبارت 
 		 	\ref{eq:6}
 		 	همان معادله ی معروف بردار و مقدار ویژه است. پس با پیدا کردن d بردار ویژه با کمترین مقدار ویژه خواهیم توانست ماتریس تبدیل $A$ را بدست آوریم. در اینجا سوالی پیش می آید که چرا مقدار بهینه
 		 	$\mu$
 		 	 را بدست نیاوردیم. نکته ی حائز اهمیت در اینجا این است که پارمتر 
 		 	$\mu$
 		 	 یک پارامتر آزاد نیست که بتوانیم آن را تخمین بزنیم بلکه 
 		 	$\mu$
 		 	 با توجه به توزیع داده ها تخمین زده می شود. بنابراین در واقعیت باید با توجه به کاربرد و با استفاده از روش ارزیابی متقابل مقدار بهینه 
 		 	$\mu$
 		 	 را بدست آوریم.
 		}
	 	\subsubsection{W-BDA}
	 	{
	 		روش 
	 		\lr{W-BDA}
	 		همانند روش 
	 		\lr{WBDA}
	 		است با این تفاوت که با در نظر گرفتن نامتوازن بودن کلاس ها در داده ها تخمین دقیق تری را برای احتمال شرطی تعریف می کند.  همانطور که می دانیم در واقعیت ما نمیتوانیم احتمال 
	 		$P(y|x)$
	 		را به صورت مستقیم تخمین بزنیم به همین دلیل با استفاده از قانون بیز به جای آن احتمال 
	 		$P(x|y)$
	 		را تخمین میزنیم. روش 
	 		\lr{W-BDA}
	 		برای تخمین احتمال پیشین داده های موجود در دامنه مبدا و هدف به ترتیب از دو ضریب 
	 		$\alpha_s$
	 		و 
	 		$\alpha_t$
	 	  استفاده می کند که با استفاده از آن ها تناسب کلاس های هر دامنه را متوازن می کند. اگر بخواهیم این مسئله را به صورت ریاضی بیان کنیم خواهیم داشت:
	 		\begin{equation}
	 		\begin{aligned}
	 		Dist({P(y_s|x}_s),{P(y_t|x}_t))= \Vert P \left( y_{s} \vert x_{s} \right)  -P \left( y_{t} \vert x_{t} \right) \Vert  _{H}^{2}\\
	 		= \bigg \Vert \frac{P \left( y_{s} \right) }{P \left( x_{s} \right) }P \left( y_{s} \vert x_{s} \right)  -\frac{P \left( y_{t} \right) }{P \left( x_{t} \right) }P \left( y_{t} \vert x_{t} \right) \bigg \Vert  _{H}^{2}\quad \quad \quad \quad \quad\\
	 		= \Vert \alpha _{s}P \left( y_{s} \vert x_{s} \right)  - \alpha _{t}P \left( y_{t} \vert x_{t} \right)  \Vert  _{H}^{2} \quad \quad \quad \quad \quad \quad \quad \quad \quad 
	 		\end{aligned}
	 		\label{eq:7}
	 		\end{equation}
	 		
	 		بنابراین با توجه به توضیحات بالا در روش 
	 		\lr{W-BDA}
	 		کافی است معادله ی 
	 		\ref{eq:4}
	 		را به معادله ی زیر تغییر دهیم:
	 		\begin{equation}
	 		\begin{aligned}
	 		min \quad tr \left( A^{T}X \left( \left( 1 - \mu \right) M_{0} + \mu \sum _{c=1\mathrm{ }}^{C}W_{c} \right) X^{T}A \right) + \lambda \Vert A \Vert_{F}^{2} \\
	 		s.t. \quad A^{T}XHX^{T}A = I, \quad  0  \leq \mu \leq  1 \quad \quad \quad \quad \quad
	 		\label{eq:8}
	 		\end{aligned}
	 		\end{equation}
	 		
	 		\begin{figure}
	 			\centering
	 			\includegraphics[scale=0.5]{images/table1.jpg}
	 			\caption{}
	 			\label{fig:5}
	 		\end{figure}
	 	
	 	به طور خلاصه الگوریتم دو روش 
	 	\lr{BDA}
	 	و
	 	\lr{W-BDA}
	 	را میتوانید در شکل
	 	\ref{fig:5}
	 	 مشاهده کنید.
	 		
	 	}
	}

	\subsection{انتقال یادگیری آسان با استفاده از ساختارهای درون دامنه
		\protect \footnote{\lr{Easy Transfer Learning By Exploiting Intra-domain Structures}}}
	{
		\begin{figure}
			\centering
			\includegraphics[scale=0.6]{images/easyTL.JPG}
			\caption{ فرآیند روش 
				\lr{\textit{EasyTL}}  }
			\label{fig:6}
		\end{figure}
		بیشتر روش‌های سنتی و عمیق در انتقال یادگیری، روش‌های پارامتری است که باید طی فرآیندی بسیار گران قیمت و وقت گیر پارامترهای آن را تنظیم کند. ارزیابی متقابل، که متداول‌ترین استراتژی برای انتخاب مدل‌ها و تنظیم پارامترها است، در انتقال یادگیری قابل استفاده نیست زیرا اغلب داده‌های برچسب خورده در دامنه هدف وجود ندارد. اگرچه روش‌های اخیر AutoML می‌تواند به طور خودکار پارامترها را از طریق هرس درخت، boosting و شبکه‌های عصبی تنظیم کنند اما آن‌ها قادر به مدیریت توزیع‌های مختلف بین دامنه‌ها نیستند و به طور معمول مدت زمان زیادی برای همگرایی لازم دارند. در مقاله
		\cite{wang2019easy}
		الگوریتمی پیشنهاد شده است که روشی ساده اما کارآمد را پیشنهاد می‌دهد که علاوه بر این‌که مشکلات ذکر شده را حل می‌کند، این امکان را فراهم می‌کند که بتوان الگوریتم یادگیری انتقال را بر روی دستگاه‌های کوچک که دارای منایع محاسباتی محدود هستند نیز اجرا کرد. اسم این روش را
		\lr{\textit{EasyTL}}
		نامیدند.
		\lr{\textit{EasyTL}} 
		قادر است بدون نیاز به انتخاب مدل و تنظیم هایپرپارامترها ، انتقال دانش را در دامنه‌ها انجام دهد.
		\lr{\textit{EasyTL}}
		با بهره برداری از ساختارهای درون دامنه، هم انتقال ویژگی‌ها و هم انتقال طبقه‌بند را به صورت غیر پارامتری می‌آموزد. برای انتقال ویژگی‌ها روشی به نام
		\lr{\textit{intra-domain alignment}} 
		و برای انتقال طبقه‌بند روشی به نام
		\lr{\textit{intra-domain programming}} 
		را معرفی کرده است(شکل 
		\ref{fig:6}
		). در ادامه به توضیح جزئیات هر کدام از این روش‌ها می‌پردازیم.
		
		\subsubsection{\lr{intra-domain programming}}
		{
			در این بخش میخواهیم بررسی کنیم که روش 
			\lr{\textit{intra-domain programming}} 
			چگونه میتواند طبقه بند دامنه مبدا را به دامنه هدف منتقل کند. قبل از اینکه به سراغ جزئیات این روش بریم، مفهومی را به نام ماتریس احتمال M معرفی می کنیم. سطر های این ماتریس تعداد کلاس ها و ستون های ان برابر با تعداد نمونه های موجود در مجموعه داده است. هر درایه از این ماتریس نشان می دهد که چقدر احتمال دارد یک نمونه متعلق به کلاس آن سطر باشد. بنابراین جمع احتمال های یک ستون در این ماتریس برابر یک است.
			\begin{figure}[h]
				\centering
				\includegraphics[scale=0.3]{images/annotation.JPG}
				\caption{یک مثال از ماتریس احتمال M}
				\label{fig:7}
			\end{figure}
			
			در روش 
			\lr{\textit{intra-domain programming}} 
			به حای اینکه مستقیما 
			
			را بیاموزد، تلاش می کند تا مارتیس احتمال M را بدست آورد. از این رو تابع هزینه به صورت زیر تعریف می شود:
			\begin{equation}
			\mathbf{J}=\ \sum_{j}^{n_t}\sum_{c}^{C}{D_{cj}}M_{cj}
			\label{eq:9}
			\end{equation}
			
			در عبارت 
			\ref{eq:9}
			ماتریس D ماتریس فاصله است که درایه ی 
			$D_{cj}$
			نشان دهنده ی فاصله ی نمونه
			$x_j^t$
			تا مرکز c امین کلاس در دامنه مبدا است. اگر 
			$h_c$
			را به عنوان مرکز c امین کلاس در دامنه مبدا در نظر بگیریم؛ در این صورت 
			$D_{cj}$
			و 
			$h_c$
			به ترتیب به صورت زیر محاسبه می شود:
			\begin{equation}
			D_{cj}\ =\ \left|\left|x_t^j\ -\ h_c\right|\right|
			\label{eq:10}
			\end{equation}
			
			\begin{equation}
			h_c=\frac{1}{\left|\mathrm{\Omega}_s^{\left(c\right)}\right|}\sum_{i}^{n_s}{x_i^s.\ \mathbb{I}\left(y_i^s=c\right)}
			\label{eq:11}
			\end{equation}
			
			اکنون برای ان که ماتریس احتمال M را بدست آوریم باید تابع هزینه معادله
			\ref{eq:9}
			را به حداقل برسانید. بدین منظور قید هایی برای این تابع نعریف می شود. اولین قید است است که جمع احتمال هر ستون از ماتریس M برابر با یک باشد. قید دومی که نعریف می شود این است که حداقل باید برای هر کلاس یک نمونه وجود داشته باشد و قید سوم هم همانطور که می دانیم مقدار هر درایه از ماتریس M باید یک عدد در بازه ی صفر تا یک باشد. بنابراین به صورت ریاضی داریم:
			\begin{equation}
			\begin{aligned}
			min\ \ \ \ \mathbf{J}=\ \sum_{j}^{n_t}\sum_{c}^{C}{D_{cj}}M_{cj} \ \ \ s.t \\
			0\le\ M_cj\le1,\ \ \ \ \ \ \ \ \ \ \ \ \ \ \ \ \ \ \ \ \ \ \ \ \ \ \ \  \\ 
			\sum_{c}^{C}M_{cj}=1,\ \forall\ j\ \epsilon\ 1,\ \ldots,\ n_t, \ \ \ \  \\ 
			\sum_{j}^{n_t}M_{cj}\geq1,\ \forall\ c\ \epsilon\ 1,\ \ldots,C  \ \ \ \ \ \ 
			\end{aligned}
			\label{eq:12}
			\end{equation}
			
			برای حل عبارت 
			\ref{eq:12}
			از پکیچ PuLP در پایتون استفاده می کنیم و در نهایت ماتریس M بدست می آید. در گام آخر برچسب نمونه
			$x_j^t$
			به شکل زیر محاسبه می گردد:
			\begin{equation}
			y_{j}^{t}=\mathop{argmax}_{r}\frac{M_{rj}}{ \sum _{c}^{C}M_{cj}}\ \ \ \ \ \ for \ \ r \in  \{ 1,  \ldots , C \}
			\label{eq:13}
			\end{equation}
			
			نکته ی قابل توجه در این روش این است که هیچ پارامتری در این روش وجود ندارد که نباز ب تنظیم کردن داشته باشد. به همین علت است که به این روش، روش غیر پارامتری می گوییم.
		}
		
		\subsubsection{\lr{intra-domain alignment}}
		{
			همانطور که در شکل 
			\ref{fig:6}
			مشخص است؛هدف از 
			\lr{\textit{intra-domain alignment}}
			تطبیق ویزگی ها بین دامنه ی مبدا و هدف است که باعث کم شدن واگرایی بین دامنه ها می شود. روش هایی زیادی برای این کار وجود دارد مانند BDA‌ و GFK . اما در اینجا ما از روش 
			\lr{\textit{CORrelation ALignment}}
			استفاده می کنیم که هم از لحاظ محاسباتی کارآمد است و هم یک روش غیر پارامتری است. فرمولی که این روش برای تطبیق دامنه ها معرفی می کند به صورت زیر است:
			\begin{equation}
				z^{r}= \bigg\{ \begin{array}{c}
				x^{r}.  \left( cov \left( x^{s} \right) +E_{s} \right) ^{-\frac{1}{2}}. \left( cov \left( x^{t} \right) +E_{t} \right) ^{\frac{1}{2}}\ \ \ \ \ if \ r=s\\
				x^{r} \ \ \ \ \ \ \ \ \ \ \ \ \ \ \ \ \ \ \ \ \ \ \ \ \ \ \ \ \ \ \ \ \ \ \ \ \ \ \ \ \ \ \ \ \ \ \ \ \ \ \ \ \ \ \ \ \ \ \ \ \ \ \ \   if \ r=t\\
				\end{array}
				\label{eq:14}
			\end{equation}
			
			\begin{figure}
				\centering
				\includegraphics[scale=0.7]{images/easyTLAlgo.JPG}
				\caption{}
				\label{fig:8}
			\end{figure}
			
			الگوریتم روش 
			\lr{\textit{EasyTl}}
			را به طور خلاصه می توانید در شکل
			\ref{fig:8}
			مشاهده کنید.
		}
	}

}